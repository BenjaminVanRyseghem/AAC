\documentclass[a4paper,10pt]{article}
\input{/Users/benjamin/Documents/Education/LaTeX/macro.tex}

\title{AAC: S�ance 2}
\author{Benjamin \bsc{Van Ryseghem}}

\begin{document}
\maketitle

\section{Exercice 1}
\subsection{Question 1}
\begin{tabular}{ r c l}
Si $n <= 2$&:& $A(n) = 1$ \\
Sinon&:&  $A(n) = A(n-1)+A(n-2)+ A(n-3)+1$\\
Donc &:&\\
&&$A(n) \geq 3A(n-3)$\\
&&$A(n) \geq 3^{\frac{n}{3}}A(0)$
\end{tabular}

On en d�duis que cet algorithme est en $\Omega(3^n)$

\subsection{Question 2}
\begin{verbatimtab}
ArrayList<Integer> data; //On suppose data suffisament grand

int T2(int n){
	int result;
	if(data[n] != null){ return data[n]);
	if (n<=2){
		data[n] = 1;
		return 1; }
	result = T2(n-1)+2*T2(n-2)+T2(n-3);
	data[n] = result;
	return result;
}
\end{verbatimtab}

\section{Exercice 2}
\subsection{Question 1}
$\binom{k}{n} = C_n^k  = \dfrac{k!}{(n-k)!n!}$ 

\subsection{Question 2}
$PL(i, j, r)$ = Cout de placer les $i$ premi�res stations entre les emplacements 0 et $j$ sachant que le i+1 est en $r$.

\paragraph{Ce que l'on cherche ?}
$PL(k, p-1, p)$

\paragraph{Ce que l'on sait ?}
$PL(i, i-1, r) = \displaystyle\sum_{j=0}^{i-2}(d_{j_a}-d_j)^2 + (d_{i-1} - d_r)^2$

\signature

\end{document}
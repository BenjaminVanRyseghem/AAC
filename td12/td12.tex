\documentclass[a4paper,10pt]{article}
\input{/Users/benjamin/Documents/Education/LaTeX/macro.tex}

\title{AAC: S�ance 12}
\author{Benjamin \bsc{Van Ryseghem}}

\begin{document}
\maketitle

\section{Exercice 1}
\small
\hspace{-1.5cm}
\begin{tabular}{|l|c|l|}
\hline
question ? & r�ponse & explication(s)\\
\hline
\multirow{2}{*}{Tout langage reconnu par un automate est r�cursif}&\multirow{2}{*}{vrai}&un automate (fini) s'arr�te toujours\\
&&lorsqu'il prend un mot en entr�e\\
$\{a^n/\mbox{n est premier}\}$ est r�cursif&vrai&on sait �crire un algorithme pour �a\\
\multirow{2}{*}{Tout langage alg�brique est r�cursif}&\multirow{2}{*}{vrai}&un langage alg�brique est reconnu par \\
&&un automate\\
Si $L_1$ et $L_2$ sont r�cursifs alors $L_1\cap L_2$ est r�cursif&vrai&parce queeeeeeeeeeeee\\
$L_1\cap L_2$ est r�cursif et $L_2$ est r�cursif alors $L_1$ est r�cursif&faux& $L_2 = \varnothing$ , $L_1$ n'importe quoi non r�cursif\\
$L_1\cup L_2$ est r�cursif et $L_2$ est r�cursif alors $L_1$ est r�cursif&faux& $L_1$ non r�cursif inclue dans $L_2$\\
L'intersection d'un langage r�cursif et d'un langage alg�brique &\multirow{2}{*}{vrai}&\multirow{2}{*}{\emph{cf.} 3 et 4}\\
est un langage r�cursif&&\\
Le compl�mentaire d'un langage r�cursif est r�cursif&vrai&On retourne \verb?!a? au lieu de \verb?a?\\
Le compl�mentaire d'un langage alg�brique est r�cursif&vrai&\emph{cf.}3 et 8\\
\hline
\end{tabular}
\normalsize

\section{Exercice 2}
Si Goldbach s'arrete

\section{Exercice 3}
to do

\section{Exercice 4}
\subsection{Question 1}
\paragraph{a}
\mbox{}

\mbox{}

\begin{tabular}{|l | c c |c| c c|}
\hline
�tat&&&v&&\\
\hline
q0&&B&a&B&\\
qR&B&B&B&&\\
q1&&B&B&B&\\
\hline
\end{tabular}

Le mot n'est pas reconnu.

\paragraph{aa}
\mbox{}

\mbox{}

\begin{tabular}{|l | c c c|c|c c c|}
\hline
�tat&&&&v&&&\\
\hline
q0&&&B&a&a&B&\\
qR&&B&B&a&B&&\\
qR&B&B&a&B&&&\\
q1&&B&B&a&B&&\\
qL&&&B&B&B&B&\\
q0&&B&B&B&B&&\\
qf&&B&B&B&B&&\\
\hline
\end{tabular}

Le mot est reconnu.

\subsection{Question 2}
Le langage est l'ensemble des mots compos�s de a de longueur pair.
\signature

\end{document}
\documentclass[a4paper,10pt]{article}
\input{/Users/benjamin/Documents/Education/LaTeX/macro.tex}
\usepackage{subscript}

\title{AAC : S�ance 7}
\author{Benjamin \bsc{Van Ryseghem}}

\begin{document}
\maketitle

\section*{Exercice 7: SAT}
\subsection*{Question 1}

forme bool�enne $\nrightarrow$  forme disjonctive $\rightarrow$ test polynomiale de satisfiabilit� $\Rightarrow$
formule bool�enne test�e en temps polynomiale.

Passer d'une forme bool�enne a une forme disjonctive ne peut pas forcement se faire de fa�on polynomiale.


\section{Exercice 1: Vrai ou Faux}
\subsection{Question 1}
\begin{tabular}{l l}
Toute propri�t� NP est aussi P & Vrai si P = NP\\
Toute propri�t� P est aussi NP & Vrai \\
Une propri�t� NP est une propri�t� non polynomiale & Faux\\
Il existe une propri�t� NP polynomiale & Vrai \\
Il existe une propri�t� NP non polynomiale & Vrai si P$\neq$NP\\
Une propri�t� NP-dur n'est pas P & Vrai si P$\neq$NP\\
\end{tabular}

\subsection{Question 2}
\begin{tabular}{l l}
Si P\textsubscript{1} se r�duit polynomialement  en Q, Q est P & Faux \\
Si Q se r�duit polynomialement  en P\textsubscript{1}, Q est P & Vrai \\
Si Q se r�duit polynomialement  en P\textsubscript{2}, Q est NP-dur & Faux \\
Si P\textsubscript{2} se r�duit polynomialement  en Q, Q est NP-dur & Vrai \\
Si Q se r�duit polynomialement  en P\textsubscript{3}, Q est NP & Vrai \\ 
Si P\textsubscript{3} se r�duit polynomialement  en Q, Q est NP & Faux \\ 
\end{tabular}

\section{Exercice 2}
\subsection{Question 1 : Montrer que le probl�me COL est NP-complet}
\paragraph{Certificat}
Un certificat pour G est juste un coloriage des noeuds. On peut par exemple le repr�senter par un tableau de couleurs index� par les sommets. On a donc :
\[
\mbox{taille du certificat} \Leftarrow \mbox{taille du graphe}
\]
(on suppose que la taille d�un graphe est au moins le nombre de sommets plus le nombre d�arcs) La taille d�un certificat est alors bien lin�aire, donc polynomialement born�e, par rapport � celle de la donn�e.

\paragraph{Algorithme de preuve}
Un certificat est valide $\Leftrightarrow$ aucun arc ne relie deux noeuds de m�me couleur: le v�rifier est bien polynomial:
\begin{verbatimtab}

boolean A(col, G){ 
	Pour chaque arc (s,d) de G
		si col(s)=col(d) retourner Faux; 
	retourner Vrai;
}
\end{verbatimtab}
La complexit� de l�algorithme est de l�ordre de card(A) donc bien polynomiale.

\paragraph{Conclusion} Le probl�me est bien NP. Il reste � montrer qu'il est NP-dur.

Soit une instance de 3-COL d�finie par  $G_3 = (S_3, A_3)$ un graphe non orient�. On construit une instance de COL de la mani�re suivante:
\begin{itemize}
\item on fixe $G=G_3$.
\item on fixe k = 3 pour l'instance de COL.
\item la transformation est bien polynomiale.
\end{itemize}

\paragraph{Montrons que $I_3 vrai \Leftrightarrow I vrai$}
Montrons que s'il existe une 3-coloration dans $G_3$ alors dans $G$ aussi et inversement.

Donc 3-COL se r�duit polynomialement vers COL.
Or comme 3-COL est NP-complet, on en d�duit que COL est NP-dur.

\paragraph{Conclusion} COL est NP et NP-dur, donc COL est NP-complet.

\subsection{Question 2: }


\signature
\end{document}
\documentclass[a4paper,10pt]{article}
\input{/Users/benjamin/Documents/Education/LaTeX/macro.tex}

\title{AAC: TP1 }
\author{Fran�ois \bsc{Lepan}\\Benjamin \bsc{Van Ryseghem}}

\begin{document}
\maketitle

\section{Erreur de r�duction}

\subsection{Calcul du minimum}
On suppose une image ayant plus de 0 pixel, \emph{i.e.} \\
\newtheorem{guess}{Hypothesis} 
\begin{guess}
\label{hyp:1}
\begin{minipage}[t]{4.5 in}
\[\forall start \in \mathbb{N}, \forall stop \in \mathbb{N}, start \leq stop, \displaystyle\sum_{i=start}^{stop}(histo[i]) \geq 0\]
\end{minipage}
\end{guess}

Soit $f_{[i, j ]}$ la fonction qui qui � un niveau de gris $x$ retourne l'erreur $f(x)$ associ�e a ce niveau de gris sur l'intervalle $[start, stop]$.

\begin{eqnarray*}
f_{[start, stop]}(x)&=&\displaystyle\sum_{i=start}^{stop}histo[i]\times(x-palette[i])^2\\
f_{[start, stop]}(x)&=&\displaystyle\sum_{i=start}^{stop}histo[i]\times(x^2-2x\times palette[i]+palette[i]^2)\\
f_{[start, stop]}'(x)&=&\displaystyle\sum_{i=start}^{stop}histo[i]\times(2x-2palette[i])\\
f_{[start, stop]}'(x)&=&2\times\displaystyle\sum_{i=start}^{stop}histo[i]\times(x-palette[i])\\
f_{[start, stop]}''(x) &=&2\times\displaystyle\sum_{i=start}^{stop}histo[i]
\end{eqnarray*}
Minimiser l'erreur revient a minimiser $f_{[start; stop]}(x)$ pour $x \geq 0$.\\
Or $f_{[i, j]}$ est convexe sur $\mathbb{R}, \forall i,j \in \mathbb{N}, i \leq j$ car $f_{[i,j]}''(x) \geq 0, \forall x \in \mathbb{R}$ donc $f$ est minimal quand $f'$ est nulle.

\begin{eqnarray*}
f_{[start, stop]}'(x)&=& 0\\
\iff 2\times\displaystyle\sum_{i=start}^{stop}histo[i]\times(x-palette[i]) &=& 0\\
\iff \displaystyle\sum_{i=start}^{stop}histo[i]\times(x-palette[i]) &=& 0\\
\iff \displaystyle\sum_{i=start}^{stop}(histo[i]\times x) - \displaystyle\sum_{i=start}^{stop}(histo[i]\times palette[i]) &=& 0\\
\iff\displaystyle\sum_{i=start}^{stop}(histo[i]\times x) &=& \displaystyle\sum_{i=start}^{stop}(histo[i]\times palette[i])\\
\iff x\times\displaystyle\sum_{i=start}^{stop}(histo[i]) &=& \displaystyle\sum_{i=start}^{stop}(histo[i]\times palette[i])\\
\end{eqnarray*}

Donc d'apres l'hypoth�se~\ref{hyp:1} :

\begin{eqnarray*}
\iff x\times\displaystyle\sum_{i=start}^{stop}(histo[i]) &=& \displaystyle\sum_{i=start}^{stop}(histo[i]\times palette[i])\\
\iff x &=& \dfrac{\displaystyle\sum_{i=start}^{stop}(histo[i]\times palette[i])}{\displaystyle\sum_{i=start}^{stop}(histo[i])}\\
\end{eqnarray*}

Donc l'erreur est minimale quand $x= \dfrac{\displaystyle\sum_{i=start}^{stop}(histo[i]\times palette[i])}{\displaystyle\sum_{i=start}^{stop}(histo[i])}$.

Appellons $x_{[i,j]}$ la valeur de $x$ tel que $f_{[i,j]}$ est minimale.

\subsection{Justification de l'�quation de r�currence}

Soit $g$ qui a un intervalle associe l'erreur minimale obtenue en codant les niveaux de l�intervalle de la palette sur une seule couleur.

Donc $g([i,j]) = f_{[i, j]}(x_{[i, j]})$.\\
Or $f$ est une somme donc associative.\\
De ce fait, \[\forall x \in \mathbb{R}, \forall i,j,k \in \mathbb{N}, i \leq j \leq k, f_{[i, k]}(x) = f_{[i, j]}(x) + f_{[j, k]}(x)\]

Selon les m�mes propri�t�s, l'erreur minimal d'un intervalle est la somme des erreurs minimales de ses sous intervalles.\\
On peut donc �crire:
\[  \forall i,j,k \in \mathbb{N}, i \leq j \leq k, g([i,k]) = g([i,j] \cup [j,k]) = g([i,j]) + g([j,k])\]

Le probl�me est de r�duire une palette de $\alpha$ niveaux de gris en une palette de $\beta$ niveaux de gris ($\beta < \alpha$) tout en minimisant l'erreur.\\
Or: 
\[ \forall \alpha \in \mathbb{N}, \exists t_0,\dots,t_{\beta-1} \in \mathbb{N}, t_0 \leq \dots \leq t_{\beta-1}\ / \ [0, \alpha-1] = \underbrace{[0, t_0] \cup \dots \cup [t_{\beta-2},t_{\beta-1}]}_\text{$\beta$ �l�ments}\]
Donc d'apr�s la propri�t� pr�c�dente:
\begin{eqnarray*}
g([0, \alpha -1]) &=& g([0, t_0] \cup \dots \cup [t_{\beta-2},t_{\beta-1}]) = g([0, t_0]) + \dots + g([t_{\beta-2},t_{\beta-1}])\\
g([0, \alpha -1]) &=&  \underbrace{g([0, t_0]) + \displaystyle\sum_{i=0}^{\beta-2}([t_i, t_{i+1}])}_\text{$\beta$ �l�ments}
\end{eqnarray*}
	
Donc trouver $g([0, \alpha -1])$ revient � trouver les $\beta$ intervalles $[0, t_0],\ \dots\ , [t_{\beta-2},t_{\beta-1}]$.


\signature[Fran�ois \bsc{Lepan}\\Benjamin \bsc{Van Ryseghem}]

\end{document}